
\prefacesection{Justificación}
La tecnología actualmente evoluciona continuamente y de forma acelerada mostrando mas dispositivos de  mayores funciones y con mayores recursos, por ende, pensar en la construcción de un circuito que promueva el uso de la medicina general en un rubro especifico para la sociedad en general no es una idea que no pase por alto, en este caso hablando en particular del EGC en cuestión, nos permite mostrar la idea que deseamos expresar, la situación actual de México es de una población de jóvenes en su mayoría, sin embargo estas cifras (Véase cifras del INEGI) se modificaran con el tiempo y en unas décadas se pasara de un país de jóvenes a un país de adultos, sin embargo, esto nos da una idea de hacia donde se dirige la situación del país y que elementos debemos tomar en cuenta para la creación de nuevos circuitos que ayuden a la población, pensando en esto la creación de un EGC que nos permita conectarse con un teléfono celular y realizar un monitoreo  del corazón, es una idea que busca el acceso de nuevas herramientas en materia de medicina para que el usuario pueda observar de manera practica su evolución sin consultar directamente una clínica que no necesariamente este dentro de su localidad.\newline\par

Por otra parte la evolución tecnológica ha sufrido cambios sustanciales con respecto a cada generación de dispositivos, esto, ha aumentado las capacidades técnicas de cada dispositivo y por ende las funciones que estos pueden realizar, a pesar de los esfuerzos que se han impuestos en estos dispositivos, el presentar nuevas herramientas para distintas funciones ha sido un reto de hoy en día para los teléfonos celulares de ultima generación debido a que los desarrolladores no están familiarizados con todas las áreas de investigación y esto abre las puertas a que ingenieros de todas las ramas puedan aportar sus conocimientos para aplicarlos de manera mas practica.